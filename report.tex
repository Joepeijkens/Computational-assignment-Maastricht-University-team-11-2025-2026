\documentclass[11pt,a4paper]{article}

% -----------------------------
% Packages
% -----------------------------
\usepackage[margin=1in]{geometry}
\usepackage{amsmath}
\usepackage{booktabs}
\usepackage{graphicx}
\usepackage{hyperref}
\usepackage{pgfplotstable}
\usepackage{siunitx}

\sisetup{
  round-mode = places,
  round-precision = 3
}

% -----------------------------
% Title info
% -----------------------------
\title{MRI Scheduling System: Bootstrap Input Modelling and Discrete-Event Simulation}
\author{(Student Name(s))}
\date{\today}

\begin{document}
\maketitle

% =============================
% Executive Summary
% =============================
\section*{Executive Summary}
This report evaluates an MRI scheduling system using historical call and scan data, bootstrap-based statistical inference, and discrete-event simulation (DES).
Two scheduling policies are compared: (i) a dedicated system where each scanner serves a fixed patient type, and (ii) a pooled system where both scanners can serve both patient types.

Patient Type~2 exhibits higher variability and an unknown distribution for both durations and arrivals; therefore, empirical (nonparametric) modelling is used for Type~2.
Simulation results (automatically read from the R output files) quantify appointment waiting times, delays, overtime, and utilization under both policies.
Based on the DES results, the pooled policy is recommended if it reduces overtime risk and/or improves utilization without materially worsening access performance.

\vspace{0.5em}
\noindent\textbf{Reproducibility note.} All tables in this report are read directly from CSV output files produced by the accompanying R scripts; no results are typed manually.

% =============================
% 1. Introduction
% =============================
\section{Introduction}
The goal of this assignment is to:
\begin{enumerate}
    \item Estimate stochastic input models for arrivals and scan durations for two patient types using appropriate bootstrap methods.
    \item Build and run a DES model that schedules patients into fixed appointment slots and evaluates system performance under uncertainty.
    \item Compare the \textit{old} dedicated-scanner policy with the \textit{new} pooled-scanner policy and recommend an operational choice.
\end{enumerate}

% =============================
% 2. Data
% =============================
\section{Data Description}
The dataset \texttt{ScanRecords.csv} contains historical records of patient calls and scan durations.
Calls occur during working hours (08:00--17:00). Scan durations are measured from historical completed scans.

We consider two patient types:
\begin{itemize}
    \item \textbf{Type 1} (distributional assumptions provided for key elements),
    \item \textbf{Type 2} (distributional form not provided; treated as unknown).
\end{itemize}

% =============================
% 3. Part 1: Statistical Input Modelling
% =============================
\section{Part 1: Statistical Input Modelling}
We model input uncertainty using bootstrap methods consistent with the plug-in principle:
parameters are treated as functionals of the underlying (unknown) distribution and estimated using the empirical distribution or a parametric model when justified.

\subsection{Type 1 scan durations: parametric bootstrap}
For Type~1 scan durations, the case assumption is a Normal distribution:
\[
X_1 \sim \mathcal{N}(\mu_1,\sigma_1^2).
\]
We estimate $\hat{\mu}_1$ and $\hat{\sigma}_1$ from the sample and compute uncertainty via a \textbf{parametric bootstrap} by resampling from
$\mathcal{N}(\hat{\mu}_1,\hat{\sigma}_1^2)$ and recomputing statistics across bootstrap replicates.

\subsection{Type 2 scan durations: empirical (EDF) + nonparametric bootstrap}
For Type~2 scan durations, the distributional form is unknown.
We therefore estimate the distribution nonparametrically using the empirical distribution function (EDF) $\hat{F}_n$.
Random variates for simulation are generated by resampling observed durations with replacement, which is equivalent to sampling from $\hat{F}_n$.
Uncertainty is quantified using a \textbf{nonparametric bootstrap}.

\subsection{Arrivals}
\textbf{Type 1.} Daily Type~1 call counts are modelled as Poisson:
\[
N_1(d)\sim \mathrm{Poisson}(\lambda_1),
\]
with $\hat{\lambda}_1$ estimated as the mean daily count. Uncertainty is obtained by bootstrapping daily counts across days.
Under a homogeneous within-day Poisson process assumption, interarrival times during working hours are exponential.

\medskip
\noindent\textbf{Type 2.} For Type~2, no parametric arrival process is imposed.
We compute within-day interarrival times from observed call times and model their distribution empirically via the EDF, with nonparametric bootstrap uncertainty.

% =============================
% 4. Insert Part 1 results from R outputs
% =============================
\section{Part 1 Results: Estimated Input Models}
Table~\ref{tab:inputs} is automatically read from \texttt{OR\_inputs\_summary.csv} generated by the R code (Option A / EDF for Type~2).

\pgfplotstableread[col sep=comma]{OR_inputs_summary.csv}\inputtable

\begin{table}[h]
\centering
\caption{Estimated stochastic input parameters (read from \texttt{OR\_inputs\_summary.csv}).}
\label{tab:inputs}
\pgfplotstabletypeset[
  columns={item,value},
  columns/item/.style={string type, column type=l, column name=Quantity},
  columns/value/.style={column type=r, column name=Estimate},
  every head row/.style={before row=\toprule, after row=\midrule},
  every last row/.style={after row=\bottomrule}
]{\inputtable}
\end{table}

\subsection{Empirical CDF visualizations (Type 2)}
Figures~\ref{fig:edf_dur} and~\ref{fig:edf_iat} visualize the EDFs used as the nonparametric ``distribution'' for Type~2.

\begin{figure}[h]
\centering
\includegraphics[width=0.80\linewidth]{EDF_Type2_Duration.png}
\caption{Empirical CDF (EDF) for Type~2 scan durations (produced by the R code).}
\label{fig:edf_dur}
\end{figure}

\begin{figure}[h]
\centering
\includegraphics[width=0.80\linewidth]{EDF_Type2_Interarrival.png}
\caption{Empirical CDF (EDF) for Type~2 interarrival times (produced by the R code).}
\label{fig:edf_iat}
\end{figure}

% =============================
% 5. Part 2: DES model
% =============================
\section{Part 2: Discrete-Event Simulation Model}
We build a discrete-event simulation of the appointment system.
Key operational rules implemented in code include:
\begin{itemize}
    \item Calls occur during working hours (08:00--17:00).
    \item No same-day scheduling (earliest appointment is the next working day).
    \item Appointments are booked in fixed slot lengths by patient type.
    \item Realized scan durations are stochastic, so actual start times may be delayed.
    \item Overtime occurs if the last scan completes after 17:00.
\end{itemize}

\subsection{Policies compared}
\begin{itemize}
    \item \textbf{Old (dedicated) policy:} Scanner 1 serves Type~1; Scanner 2 serves Type~2.
    \item \textbf{New (pooled) policy:} Both scanners can serve both types; each call is scheduled to the earliest feasible slot across scanners.
\end{itemize}

\subsection{Performance measures}
The simulation reports:
\begin{itemize}
    \item \textbf{Appointment waiting time (working hours):} time from call to scheduled start (compressed to working time).
    \item \textbf{Day-of delay (minutes):} actual start time minus scheduled start time.
    \item \textbf{Overtime (minutes per machine-day):} time after 17:00.
    \item \textbf{Utilization:} busy time divided by (scheduled day length + overtime).
\end{itemize}

% =============================
% 6. Insert DES results from R outputs
% =============================
\section{Part 2 Results: Simulation Outcomes}
Table~\ref{tab:des} is automatically read from \texttt{part2\_sim\_summary.csv} produced by the DES code.

\pgfplotstableread[col sep=comma]{part2_sim_summary.csv}\destable

\begin{table}[h]
\centering
\caption{Discrete-event simulation results comparing old vs pooled policies (read from \texttt{part2\_sim\_summary.csv}).}
\label{tab:des}
\pgfplotstabletypeset[
  columns={Policy,ApptWait_mean_h,ApptWait_p95_h,Delay_mean_min,OT_mean_min,OT_p95_min,OT_prob_gt0,Util_mean},
  columns/Policy/.style={string type, column type=l, column name=Policy},
  columns/ApptWait_mean_h/.style={column name=Mean wait (h)},
  columns/ApptWait_p95_h/.style={column name=P95 wait (h)},
  columns/Delay_mean_min/.style={column name=Mean delay (min)},
  columns/OT_mean_min/.style={column name=Mean OT (min)},
  columns/OT_p95_min/.style={column name=P95 OT (min)},
  columns/OT_prob_gt0/.style={column name=P(OT$>$0)},
  columns/Util_mean/.style={column name=Utilization},
  every head row/.style={before row=\toprule, after row=\midrule},
  every last row/.style={after row=\bottomrule}
]{\destable}
\end{table}

\subsection{Distributional comparisons}
Figures~\ref{fig:wait} and~\ref{fig:overtime} show the distributional outcomes for appointment waiting time and overtime.

\begin{figure}[h]
\centering
\includegraphics[width=0.85\linewidth]{fig_sim_wait.png}
\caption{Appointment waiting time distribution (working hours): old vs pooled.}
\label{fig:wait}
\end{figure}

\begin{figure}[h]
\centering
\includegraphics[width=0.85\linewidth]{fig_sim_overtime.png}
\caption{Overtime distribution (minutes per machine-day): old vs pooled.}
\label{fig:overtime}
\end{figure}

% =============================
% 7. Discussion
% =============================
\section{Discussion}
The results reflect the operational impact of variability in both demand and scan durations.
Type~2 is modelled empirically because its distribution is unknown; this preserves heavy tails and day-to-day fluctuations observed in the data.
Pooling scanners can mitigate variability by allowing dynamic workload sharing across machines.

In interpreting Table~\ref{tab:des}, the policy choice depends on the trade-off between:
\begin{itemize}
    \item \textbf{Access performance} (appointment waiting time and delays), and
    \item \textbf{Operational risk} (overtime probability and overtime tail behavior).
\end{itemize}

% =============================
% 8. Recommendation
% =============================
\section{Managerial Recommendation}
Based on the DES results, we recommend the policy that achieves the best overall balance between access performance and overtime risk.
If the pooled policy yields lower mean and tail overtime and comparable (or improved) appointment waiting times, then pooling is preferred because it improves robustness against variability.

If the pooled policy improves overtime but significantly worsens appointment waiting times, then a hybrid recommendation is appropriate (e.g., pooling with revised slot lengths or additional scheduling constraints).
The simulation framework implemented here can be used to evaluate such refinements.

% =============================
% 9. Sensitivity and robustness
% =============================
\section{Sensitivity and Robustness}
The simulation is run over multiple replications and a multi-day horizon.
Using EDF-based modelling for Type~2 reduces dependence on arbitrary parametric assumptions and provides robustness with respect to distributional misspecification.
A recommended extension is to propagate input uncertainty (from Part~1 bootstrap distributions) into Part~2 by re-running the DES across bootstrap-resampled inputs and comparing the resulting distribution of KPIs.

% =============================
% 10. Assumptions and limitations
% =============================
\section{Assumptions and Limitations}
This model abstracts from several real-world factors:
\begin{itemize}
    \item No cancellations or no-shows are modelled.
    \item No urgent add-on cases are modelled.
    \item Slot lengths are fixed by type and do not dynamically adapt to congestion.
    \item The system is simulated under the working-hours calendar defined in the case; non-working time is treated as no progression for appointment waiting time calculations (working-time scale).
\end{itemize}
Despite these limitations, the model captures the primary drivers of overtime and delay: demand variability and stochastic service durations.

% =============================
% 11. Reproducibility
% =============================
\section{Reproducibility}
All results and figures are produced using R scripts included with the submission.
The report reads results directly from:
\begin{itemize}
    \item \texttt{OR\_inputs\_summary.csv} (Part 1 input modelling),
    \item \texttt{part2\_sim\_summary.csv} (Part 2 DES summary),
    \item \texttt{EDF\_Type2\_Duration.png}, \texttt{EDF\_Type2\_Interarrival.png}, \texttt{fig\_sim\_wait.png}, \texttt{fig\_sim\_overtime.png}.
\end{itemize}
Re-running the scripts reproduces all tables and figures without manual editing.

% =============================
% 12. Final remarks
% =============================
\section{Final Remarks}
This assignment combines bootstrap inference with discrete-event simulation to evaluate an MRI appointment system under uncertainty.
The use of empirical (EDF) inputs for unknown distributions provides a data-driven modelling approach, while simulation quantifies operational trade-offs between access and overtime.
The resulting framework supports evidence-based policy selection and can be extended to test alternative slot lengths and scheduling rules.

\end{document}
